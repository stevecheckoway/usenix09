% \iffalse The license starting three lines down applies to this file
%<*batchfile>
\begingroup\input docstrip \preamble
Copyright (C) 2009-2011 by Stephen Checkoway <s@cs.ucsd.edu>

This file may be distributed and/or modified under the
conditions of the LaTeX Project Public License, either
version 1.3c of this license or (at your option) any later
version.  The latest version of this license is in:

    http://www.latex-project.org/lppl.txt

and version 1.3c or later is part of all distributions of
LaTeX version 2005/12/01 or later.

This work has the LPPL maintenance status 'maintained'.

The Current Maintainer of this work is Stephen Checkoway.

This work consists of usenix09.dtx and the derived file
usenix09.cls.
\endpreamble
\keepsilent
\usedir{tex/latex/\jobname}
\askforoverwritefalse
\generate{\file{\jobname.cls}{\from{\jobname.dtx}{}}}
\endgroup

\documentclass{ltxdoc}
\usepackage{mathptmx,courier}
\usepackage[scaled=.92]{helvet}
\usepackage[T1]{fontenc}
\usepackage[breaklinks=true,pdfborder={0 0 0}]{hyperref}

\CheckSum{628}
\CharacterTable
 {Upper-case    \A\B\C\D\E\F\G\H\I\J\K\L\M\N\O\P\Q\R\S\T\U\V\W\X\Y\Z
  Lower-case    \a\b\c\d\e\f\g\h\i\j\k\l\m\n\o\p\q\r\s\t\u\v\w\x\y\z
  Digits        \0\1\2\3\4\5\6\7\8\9
  Exclamation   \!     Double quote  \"     Hash (number) \#
  Dollar        \$     Percent       \%     Ampersand     \&
  Acute accent  \'     Left paren    \(     Right paren   \)
  Asterisk      \*     Plus          \+     Comma         \,
  Minus         \-     Point         \.     Solidus       \/
  Colon         \:     Semicolon     \;     Less than     \<
  Equals        \=     Greater than  \>     Question mark \?
  Commercial at \@     Left bracket  \[     Backslash     \\
  Right bracket \]     Circumflex    \^     Underscore    \_
  Grave accent  \`     Left brace    \{     Vertical bar  \|
  Right brace   \}     Tilde         \~}


\EnableCrossrefs
\CodelineIndex
\CodelineNumbered
\RecordChanges
\usepackage{trace}
\begingroup
    \def\ProvidesClass#1 [#2]{\endinput\gdef\temp{\ProvidesFile{#1.dtx}[#2]}}
    \input\jobname.cls
\endgroup
\temp
\GetFileInfo{\jobname.dtx}
\title{The \textsf{USENIX09} class}
\author{Stephen Checkoway\\ \texttt{s@cs.ucsd.edu}}
\date{Version \fileversion, \filedate, Generated \today}

\begin{document}
\maketitle

\phantomsection
\addcontentsline{toc}{section}{\abstractname}
\begin{abstract}
This class is designed to meet the basic requirements for USENIX 2009
paper submission.%
\footnote{\url{http://www.usenix.com/events/usenix09/instrux/details.html}}
It should be used in place of the \textsf{article} class.
\end{abstract}

\phantomsection
\addcontentsline{toc}{section}{\contentsname}
\tableofcontents

\section{Introduction}
This package is meant to simplify complying with the USENIX 2009 paper
submission format by including the appropriate packages and setting
document parameters to values appropriate for USENIX 2008 proceedings.
USENIX provides an example style file that was designed for USENIX
2005. It contains some of the basic functionality of this class, but
much is left up to the user. In addition, the sample file provided
does not even make use of all of the functionality provided.

\section{Usage}
The |usenix09| document class should be used in place of
\textsf{article}. It accepts all of the options that \textsf{article}
does as well as a number of options to control its behavior. For
initial submission, one might use,
\begin{center}
|\documentclass[twocolumn,pagenumbers]{usenix09}|
\end{center}
whereas for camera-ready, the following is more appropriate.
\begin{center}
|\documentclass[twocolumn,footasend]{usenix09}|
\end{center}

In particular, since USENIX requires two columns, the |twocolumn|
option should be used. However, since it may be desirable to have a
single column, or if two columns are to be achieved some other way,
for example using \textsf{multicol}, an option |onecolumn| may be
passed.  It is an error to not specify either |onecolumn| or
|twocolumn|.

Some options like |titlepage| may not work. The |10pt| and
|letterpaper| options are passed by default. The various options and
their rationales are described next.

\subsection{Options}
All of the options except for |twocolumn| and |onecolumn| are
independent and can be used in any combination.
\begin{description}
\item{|onecolumn|} Overrides the two column requirement and forces a
single column of text. This can be used for initial submission or when
the two columns are being set by a package rather than using the
|article| document class's option.
\item{|twocolumn|} In general, this option should be passed to
|usenix09| as USENIX requires two columns. Passing this option will
also ensure that the columns are separated by the appropriate amount
as specified by USENIX.
\item{|tight|} If your document is too long for the page constraints,
this option will remove much of the blank space left by |\maketitle|
as well as supplying the |compact| option to \textsf{titlesec}.  The
macros in \textsf{titlesec} can be manipulated to change the spacing
whether or not the |tight| option is passed.
\item{|titlecompat|} This option emulates the spacing of the
\textsf{usenix} package. It overrides |tight|, but only for
|\maketitle|.
\item{|footasend|} USENIX requires that endnotes be used in place of
footnotes. This option simplifies the conversion process by changing
|\footnote| to |\endnote| and using |\theendnotes| at the very end of
the document. Without the option, |\endnote| should be used and
|\theendnotes| should be placed at the end manually.
\item{|pagenumbers|} USENIX requires that initial submissions contain
page numbers but that the camera-ready submission contain no page
numbers.
\item{|nofonts|} By default, \textsf{usenix09} loads the
\textsf{mathptmx}, \textsf{helvet}, and \textsf{courier} packages to
set the roman and math fonts to times, the sans serif font to a scaled
down Helvetica, and the typewriter font to Courier. Pass the |nofonts|
option if you wish to set the fonts yourself. Passing this option also
sets |nofontenc|.
\item{|nofontenc|} Pass this option if you do not want T1 fonts used.
I can't think of a reason to use this, but since this isn't required
by USENIX, I thought I'd give an option for it.
\item{|noflushend|} The USENIX specifications do not say if the
columns on the final page should be the same length or not. I think it
looks better when they are the same length and thus the
\textsf{flushend} package is used by default. If you prefer to handle
this yourself, pass this option.
\end{description}

\subsection{Authors}
USENIX is particular about the format of the authors. That is, each
author's name should be set in 12~pt roman font while the institution
should be set in 12~pt italics font. The standard |\author| macro does
not accommodate this without resorting to various hacks. For example,
if you want multiple rows of authors, then rather than separate each
author by |\and|, you have to use the fact that |\author| is typeset
in a |tabular| and use two authors in a column separated by |\and|
followed by another two authors, and so forth. This doesn't scale
particularly well and it requires a lot of manual futzing.

\DescribeMacro\addauthor
To avoid the problems with |\author|, a new command
\begin{center}
|\addauthor|\oarg{footnote}\marg{name}\marg{institution}
\end{center}
is provided. It should be used once for each author. The
\meta{footnote} optional argument can be used to specify a footnote
for the author's name---as specified by \meta{name}. The
\meta{institution} argument will be below the author and in italics.

Using |\footnote| or |\thanks| in the \meta{name} or
\meta{institution} arguments directly is supported, but the order of
the footnotes will not be what you might expect.  Namely, they will
appear after footnotes in the title as well as after the
\meta{footnote} footnotes. Worse, the order will not be left-to-right,
top-to-bottom (unless there is only a single row of authors).
Basically, just don't do that. Use \meta{footnote} instead.

\DescribeMacro\author 
\DescribeMacro\and
In addition to |\addauthor|, |\author| is provided as an alias for
|\addauthor|. The |\and| macro is no longer relevant and its use is an
error.

\DescribeMacro\authorsep
\DescribeMacro\authorskip
Two parameters control the horizontal and vertical spacing of the
authors. The horizontal spacing is controlled by |\authorsep| and the
default value is that used by the \textsf{article} class's |\and|
macro. The vertical spacing is controlled by |\authorskip|.  The
default value is probably a reasonable one. Neither of these need to
be changed, but both can be.

\DescribeMacro\authorcolumns
The number of columns of authors (and hence the number of rows) is
controled by the |\authorcolumns| parameter. If it is positive, then
the author list will be typeset using that many columns. If the number
of columns specified cannot fit on the page, then the columns will
\emph{not} wrap. Instead, the authors will extend into the margins.

If |\authorcolumns| is nonpostive, then the default number of columns
will be chosen. If there are one to four authors (inclusive), then
they will be in a single row. Five and six authors use two rows of at
most three columns. Seven and eight authors use two rows of at most
four columns. Nine authors use three rows of three columns. Ten or
more authors are typeset in rows of at most four columns. In all
cases, only the last row may have fewer than the maximum number of
columns.

If the authors names and institutions are very long, they may not fit
into the number of columns chosen by default. In this case,
|\authorcolumns| should be manually set or |\authorsep| can be
decreased.

\DescribeMacro\authornewline
\DescribeMacro\institutionnewline
Additional lines can be added under the author or institution by using
|\authornewline| or |\institutionnewline|, but since the author list
is typeset in columns rather than rows, each author in a row should
have the same number of lines, even if they are blank.

In some cases with many authors and few institutions, it takes up less
space to list all authors from the first institution together with the
institution and then below it all authors from the second institution
and so on. In this case, |\authorcolumns| should be set to~1 and all
authors for a single institution should be given in the \marg{name}
parameter to |\addauthor|. Extra lines of authors for a given
institution can be given with |\authornewline|.

\StopEventually{
    \typeout{**************************************************}
    \typeout{*}
    \typeout{* To finish the installation, you have to move the}
    \typeout{* following file into a directory searched by TeX:}
    \typeout{*}
    \typeout{* \space\space \jobname.sty}
    \typeout{*}
    \typeout{* Documentation is in \jobname.\ifpdf pdf\else dvi\fi.}
    \typeout{*}
    \typeout{* Happy TeXing!}
    \typeout{**************************************************}
    \end{document}
}
\clearpage
\DocInput{\jobname.dtx}
\clearpage
\phantomsection
\addcontentsline{toc}{section}{Change History}%
\PrintChanges
\phantomsection
\addcontentsline{toc}{section}{Index}%
\PrintIndex
\Finale
%</batchfile>
% \fi
% \DoNotIndex{\newcommand,\newenvironment,\renewcommand,
%   \@date,\@empty,\@plus,\@textsuperscript,\@undefined,
%   \\,\ ,\AtBeginDocument,\AtEndDocument,\begin,
%   \begingroup,\count@,\csname,\date,\def,\divide,\edef,
%   \else,\end,\endcsname,\endgroup,\expandafter,\fi,
%   \global,\hb@xt@,\hss,\ifcase,\ifnum,\ifx,\let,
%   \lineskip,\long,\loop,\multiply,\newcount,\newif,
%   \newpage,\newskip,\noindent,\normalfont,\normalsize,
%   \null,\or,\par,\protect,\protected@edef,\protected@xdef,
%   \providecommand,\relax,\repeat,\rlap,\the,\vskip,\z@}
% \DoNotIndex{\advance,\bfseries,\c@footnote,\ClassError,
%   \columnsep,\count,\CurrentOption,\DeclareOption,\endnote,
%   \ExecuteOptions,\fontsize,\footnote,\footnotemark,
%   \footnotetext,\geometry,\hfil,\hskip,\if@twocolumn,\ifodd,
%   \itshape,\Large,\large,\LoadClass,\MessageBreak,\@ne,\pagestyle,
%   \parindent,\PassOptionsToClass,\PassOptionsToPackage,
%   \ProcessOptions,\r,\RequirePackage,\selectfont,
%   \setcounter,\setlength,\stepcounter,\string,\textwidth,
%   \theendnotes,\thefootnote,\thetitle,\thispagestyle,\thr@@,
%   \titleformat,\toks@,\tw@,\twocolumn,\upshape}
%
% \changes{v1.0}{2009/06/03}{Initial version}
% \changes{v1.1}{2009/06/04}{Exclude macros from index}
% \changes{v1.2}{2009/06/04}{User-specified number of columns}
% \changes{v1.3}{2011/01/29}{Reorganize dtx}
% \changes{v1.3}{2011/01/29}{Conserve tokens}
%
% \section{Implementation}
% \subsection{Initial code}
% First we add the package identification and then we declare some
% conditionals which we will set with options.
%    \begin{macrocode}
\NeedsTeXFormat{LaTeX2e}[1999/12/01]
\ProvidesClass{usenix09}
    [2011/01/29 v1.3 USENIX09 formatting]
%    \end{macrocode}
% \begin{macro}{\ifusenix@onecolumn}
% One column flag.
%    \begin{macrocode}
\newif\ifusenix@onecolumn \usenix@onecolumnfalse
%    \end{macrocode}
% \end{macro}
% \begin{macro}{\ifusenix@nofonts}
% Don't load any fonts.
%    \begin{macrocode}
\newif\ifusenix@nofonts \usenix@nofontsfalse
%    \end{macrocode}
% \end{macro}
% \begin{macro}{\ifusenix@nofontenc}
% No \textsf{fontenc} flag.
%    \begin{macrocode}
\newif\ifusenix@nofontenc \usenix@nofontencfalse
%    \end{macrocode}
% \end{macro}
% \begin{macro}{\ifusenix@tight}
% Tight flag.
%    \begin{macrocode}
\newif\ifusenix@tight \usenix@tightfalse
%    \end{macrocode}
% \end{macro}
% \begin{macro}{\ifusenix@titlecompat}
% |\maketitle| compatibility with \textsf{usenix} package.
%    \begin{macrocode}
\newif\ifusenix@titlecompat \usenix@titlecompatfalse
%    \end{macrocode}
% \end{macro}
% \begin{macro}{\ifusenix@pagenumbers}
% Page numbers flag.
%    \begin{macrocode}
\newif\ifusenix@pagenumbers \usenix@pagenumbersfalse
%    \end{macrocode}
% \end{macro}
% \begin{macro}{\ifusenix@noflushend}
% No \textsf{flushend} flag.
%    \begin{macrocode}
\newif\ifusenix@noflushend \usenix@noflushendfalse
%    \end{macrocode}
% \end{macro}
% \subsection{Options}
% We want to support several options which change the way the document
% is laid out.
%
% The |twocolumn| option is a global option. USENIX 2009 requires
% documents to be typeset in two columns. Pass onecolumn to allow
% single column text.
%    \begin{macrocode}
\DeclareOption{onecolumn}{\usenix@onecolumntrue}
%    \end{macrocode}
%
% Set the section spacing to be very tight for long papers.
% \changes{v1.2.2}{2009/06/05}{Use compact}
%    \begin{macrocode}
\DeclareOption{tight}{\usenix@tighttrue\PassOptionsToPackage{compact}{titlesec}}
%    \end{macrocode}
%
% Turn on |\maketitle| compatibility with \textsf{usenix} package.
% \changes{v1.2.7}{2010/03/26}{Add titlecompat}
%    \begin{macrocode}
\DeclareOption{titlecompat}{\usenix@titlecompattrue}
%    \end{macrocode}
%
% Make |\footnote| synonymous with |\endnote|.
%    \begin{macrocode}
\DeclareOption{footasend}{
    \AtBeginDocument{\let\footnote\endnote}
    \AtEndDocument{\theendnotes}}
%    \end{macrocode}
%
% Do not set the fonts at all.
%    \begin{macrocode}
\DeclareOption{nofonts}{\usenix@nofontstrue\ExecuteOptions{nofontenc}}
%    \end{macrocode}
%
% Do not load the \textsf{fontenc} package.
%    \begin{macrocode}
\DeclareOption{nofontenc}{\usenix@nofontenctrue}
%    \end{macrocode}
%
% Add page numbers.
%    \begin{macrocode}
\DeclareOption{pagenumbers}{\usenix@pagenumberstrue}
%    \end{macrocode}
%
% Do not flush end.
%    \begin{macrocode}
\DeclareOption{noflushend}{\usenix@noflushendtrue}
%    \end{macrocode}
%
% Pass all unknown options to the article class.
%    \begin{macrocode}
\DeclareOption*{\PassOptionsToClass{\CurrentOption}{article}}
%    \end{macrocode}
%
% Process those options! (Use |\relax| to prevent the search for an
% asterisk.) Load the |article| class.
%    \begin{macrocode}
\ProcessOptions\relax
\LoadClass[10pt,letterpaper]{article}
%    \end{macrocode}
%
% \subsection{Loading the packages}
% We need to load a number of packages to ensure that we conform. But
% before we do that, load the |fixltx2e| package to fix the float
% placement, among other things.
%
%    \begin{macrocode}
\RequirePackage{fixltx2e}
%    \end{macrocode}
%
% Flush end?
%    \begin{macrocode}
\ifusenix@noflushend\else
    \RequirePackage{flushend}
\fi
%    \end{macrocode}
%
% We need to set the roman and math fonts to Times, the sans serif
% font to Helvetica, and the typewriter font to Courier.
% \changes{v1.2.4}{2009/06/17}{Use mathptmx}
% \changes{v1.2.5}{2009/06/17}{Use helvet and courier}
%    \begin{macrocode}
\ifusenix@nofonts\else
    \RequirePackage{mathptmx}
    \RequirePackage[scaled=.92]{helvet}
    \RequirePackage{courier}
\fi
%    \end{macrocode}
%
% If |nofontenc| was given, do not load the package. Otherwise, use T1
% fonts.
%    \begin{macrocode}
\ifusenix@nofontenc\else
    \RequirePackage[T1]{fontenc}
\fi
%    \end{macrocode}
%
% We also want the \textsf{geometry}, \textsf{titlesec},
% \textsf{caption}, and \textsf{endnotes} packages to set the page
% layout, section heading, captions, and endnotes appropriately.
% \changes{v1.1}{2009/06/04}{Use the geometry package.}
% \changes{v1.2.6}{2009/06/22}{Don't change indentation for notes.}
%    \begin{macrocode}
\RequirePackage[letterpaper,margin=1in,nohead,columnsep=.25in]{geometry}
\RequirePackage{titlesec}
\RequirePackage[margin=10pt,font=small,labelfont=bf]{caption}
\RequirePackage{endnotes}
\let\enotesize\normalsize
\def\enoteformat{%
    \rightskip\z@
    \leftskip\z@
    \leavevmode\llap{\makeenmark}%
}
%    \end{macrocode}
%
% \subsection{Satisfying the format}
% If we don't have page numbers, then we don't need space for them.
%    \begin{macrocode}
\ifusenix@pagenumbers\else
\geometry{nofoot}
\fi
%    \end{macrocode}
%
% \begin{macro}{\usenix@twocolumn}
% Make sure we're either in |twocolumn| or |onecolumn|. Camera ready
% requires a column separation of 0.25~in.
%    \begin{macrocode}
\begingroup
\def\usenix@twocolumn{
    \ifusenix@onecolumn
        \ClassError{usenix09}{Cannot specify both twocolumn
        and onecolumn.}{At most one of twocolumn and onecolumn
        may be specified.}
    \fi}
%    \end{macrocode}
% \end{macro}
% \begin{macro}{\usenix@onecolumn}
%    \begin{macrocode}
\def\usenix@onecolumn{
    \ifusenix@onecolumn\else
        \ClassError{usenix09}{USENIX 2009 requires 2
        columns.\MessageBreak
        Pass option onecolumn to override.}{Single column
        requires the use of option onecolumn.}
    \fi}
%    \end{macrocode}
% \end{macro}
%    \begin{macrocode}
\if@twocolumn
    \usenix@twocolumn
    \setlength\columnsep{0.25in}
\else
    \usenix@onecolumn
\fi
\endgroup
%    \end{macrocode}
%
% Do we want page numbers?
%    \begin{macrocode}
\ifusenix@pagenumbers
    \pagestyle{plain}
\else
    \pagestyle{empty}
\fi
%    \end{macrocode}
%
% \begin{macro}{\maketitle}
% We need to change the title layout. This is mostly copied from
% |classes.dtx|.
%    \begin{macrocode}
\renewcommand\maketitle{\par
    \begingroup
    \renewcommand\thefootnote{\@fnsymbol\c@footnote}%
    \def\@makefnmark{\rlap{\@textsuperscript{\normalfont\@thefnmark}}}%
    \long\def\@makefntext##1{\parindent1em\noindent
        \hb@xt@1.8em{%
            \hss\@textsuperscript{\normalfont\@thefnmark}}##1}%
    \if@twocolumn
        \twocolumn[\@maketitle]%
    \else
        \newpage
        \global\@topnum\z@
        \@maketitle
    \fi
%    \end{macrocode}
% Make sure we don't use page numbers unless they were requested.
%    \begin{macrocode}
    \ifusenix@pagenumbers
        \thispagestyle{plain}%
    \else
        \thispagestyle{empty}%
    \fi
    \@thanks
    \endgroup
    \setcounter{footnote}{0}%
%    \end{macrocode}
% Now clear out a number of macros related to the title generation.
%    \begin{macrocode}
    \begingroup
    \def\r##1{\global\let##1\@undefined}%
    \global\let\@thanks\@empty
    \global\let\@author\@empty
    \global\let\@date\@empty
    \global\let\@title\@empty
    \r\thanks
    \r\maketitle
    \r\@maketitle
    \r\title
    \r\author
    \r\date
    \r\and
    \r\addauthor
    \r\usenix@authorname
    \r\usenix@institutionname
    \r\usenix@extract
    \r\usenix@temp
    \r\usenix@layoutauthorcolumn
    \r\usenix@l@youtauthorcolumn
    \r\usenix@layoutauthors
    \r\authornewline
    \r\institutionnewline
    \global\usenix@currauthor\z@
    \loop\ifnum\count@<\usenix@numauthors
        \expandafter
        \r\csname usenix@author\the\usenix@currauthor\endcsname
        \global\advance\usenix@currauthor\@ne
    \repeat
    \endgroup
}
%    \end{macrocode}
% \end{macro}
% 
% \begin{macro}{\@maketitle}
%  No date is displayed, the title is in 14~pt \textbf{bold}. The
%  author names and affiliations are handled by
%  |\usenix@layoutauthors|, below.
%    \begin{macrocode}
\renewcommand\@maketitle{%
    \newpage
    \ifusenix@titlecompat
        \vbox to2.5in{
            \vskip2em plus1fill
            \begin{center}%
                \let\footnote\thanks
                {\Large\bfseries\@title\par}%
                \vskip.375in minus.3in
                {\large\lineskip.5em \usenix@layoutauthors}%
            \end{center}%
            \par
            \vfill
        }%
    \else % usenix@titlecompat
        \null
        \ifusenix@tight\else
            \vskip2em
        \fi
        \begin{center}%
            \let\footnote\thanks
            {\Large\bfseries\@title\par}%
            \ifusenix@tight
                \vskip.75em
            \else
                \vskip1.5em
            \fi
            {\large\lineskip.5em \usenix@layoutauthors}%
            \ifusenix@tight
                \vskip.5em
            \else
                \vskip1em
            \fi
        \end{center}%
        \par
        \ifusenix@tight
            \vskip.75em
        \else
            \vskip1.5em
        \fi
    \fi % usenix@titlecompat
}
%    \end{macrocode}
% \end{macro}
%
% \subsection{Handling multiple authors}
% The author layout is is a bit tricky. We want to emulate the
% behavior of |\and| between authors, but we also need to fulfill the
% requirements of USENIX, namely the author name needs to be in roman
% font and the institution must be in italics.
%
% \begin{macro}{\and}
% To handle an arbitrary number of authors, we remove the |\and|
% macro and replace it with |\addauthor| which is to be called
% multiple times, one for each author.
%    \begin{macrocode}
\renewcommand\and{\ClassError{usenix09}{\string\and\ is not supported.
Use \string\addauthor.}{\string\and\ does not work with usenix09.}}
%    \end{macrocode}
% \end{macro}
% \begin{macro}{\addauthor}
% Save the author for later.
%    \begin{macrocode}
\newcommand\addauthor[3][]{%
    \expandafter\def\csname usenix@author\the\usenix@numauthors\endcsname{%
        {#1}{#2}{#3}}
    \global\advance\usenix@numauthors\@ne
}
%    \end{macrocode}
% \end{macro}
% \begin{macro}{\author}
% Make |\author| an alias for |\addauthor|.
%    \begin{macrocode}
\global\let\author\addauthor
%    \end{macrocode}
% \end{macro}
%
% Set the three user-accessible spacing registers to reasonable values.
%
% \begin{macro}{\authorsep}
% This controls horizontal separation.
%    \begin{macrocode}
\newskip\authorsep \authorsep1em\@plus.17fil
%    \end{macrocode}
% \end{macro}
% \begin{macro}{\authorskip}
% This controls vertical skip.
%    \begin{macrocode}
\newskip\authorskip \authorskip.5em
%    \end{macrocode}
% \end{macro}
% \begin{macro}{\authorcolumns}
% Let the user specify the number of columns desired.
%    \begin{macrocode}
\newcount\authorcolumns \authorcolumns\z@
%    \end{macrocode}
% \end{macro}
%
% Now set up some internally needed counters and macros.
%    \begin{macrocode}
\newcount\usenix@numauthors
\newcount\usenix@currauthor
\global\usenix@numauthors\z@
\providecommand\@firstofthree[3]{#1}
\providecommand\@secondofthree[3]{#2}
\providecommand\@thirdofthree[3]{#3}
%    \end{macrocode}
%
% We need to be able to extract the three components of the current
% author. The argument should be one of |\@firstofthree|,
% |\@secondofthree|, or |\@thirdofthree|. The three |\expandafter|s
% cause the constructed control sequence to be expanded before the
% argument extracts the individual component. Note that there is zero
% error checking here.
%    \begin{macrocode}
\newcommand\usenix@extract[1]{\expandafter\expandafter\expandafter
    #1\csname usenix@author\the\usenix@currauthor\endcsname}
%    \end{macrocode}
%
% To simplify matters, provide a single macro that expands to the
% author's name with a footnote if desired in roman as well as one for
% the institution.
%    \begin{macrocode}
\newcommand\usenix@authorname{%
    \protected@edef\usenix@temp{\usenix@extract\@firstofthree}%
    \begingroup
    \upshape
    \usenix@extract\@secondofthree
    \endgroup
    \ifx\usenix@temp\@empty\else
        \footnotemark[\usenix@temp]%
    \fi
}
\newcommand\usenix@institutionname{\usenix@extract\@thirdofthree}
%    \end{macrocode}
%
% \begin{macro}{\usenix@layoutauthors}
% This is the hard part. We want a somewhat unified style for the
% authors. The defaults should be sensible, but can be overwritten by
% changing |\authorcolumns|.
%    \begin{macrocode}
\newcommand\usenix@layoutauthors{%
%    \end{macrocode}
% We need to process the footnotes at this point to ensure that they
% follow any in the title.
%    \begin{macrocode}
    \global\usenix@currauthor\z@
    \loop\ifnum\usenix@currauthor<\usenix@numauthors
        \protected@edef\usenix@temp{\usenix@extract\@firstofthree}%
        \ifx\usenix@temp\@empty\else
            \stepcounter{footnote}%
            \protected@xdef\@thefnmark{\thefootnote}%
            \protected@xdef\@thanks{\@thanks
                \protect\footnotetext[\the\c@footnote]{\usenix@temp}}%
            \def\usenix@temp{\the\c@footnote}%
        \fi
%    \end{macrocode}
% We want to replace the first component of the author with the
% footnote number. We do this by expanding the |\csname| and then the
% resultant control sequence, and then the |\@gobble| to remove the
% first component and stuff the remainder into the token register.
% Then the author is redefined to be the new triple.
%    \begin{macrocode}
        \toks@\expandafter\expandafter\expandafter\expandafter\expandafter
            \expandafter\expandafter{\expandafter\expandafter\expandafter
            \@gobble\csname usenix@author\the\usenix@currauthor\endcsname}
        \expandafter\edef
            \csname usenix@author\the\usenix@currauthor\endcsname{%
                {\usenix@temp}\the\toks@}
        \global\advance\usenix@currauthor\@ne
    \repeat
%    \end{macrocode}
% Get on with laying out the authors in columns.
% \changes{v1.1}{2009/06/04}{Make PackageError into a ClassError}
%    \begin{macrocode}
    \ifnum\usenix@numauthors=\z@
        \ClassError{usenix09}{At least one author is required.}{Use
        \string\addauthor\ to add an author.}
    \fi
    \ifnum\authorcolumns>\z@
        \ifnum\authorcolumns>\usenix@numauthors
            \authorcolumns\usenix@numauthors
        \fi
    \else
        \authorcolumns
        \ifcase\usenix@numauthors
        \or \@ne %1
        \or \tw@ %2
        \or \thr@@ %3
        \or 4 %4
        \or \thr@@ %5
        \or \thr@@ %6
        \or 4 %7
        \or 4 %8
        \or \thr@@ %9
        \else 4 % How many authors do you have??
        \fi
    \fi
%    \end{macrocode}
% Now that |\authorcolumns| has been set to a reasonable value, we can
% perform the actual layout. This works by computing $\lfloor
% \mathit{numauthors}/\mathit{authorcolumns}\rfloor\cdot
% \mathit{authorcolumns}$ and laying out that many authors in the
% requested number of columns. If $\mathit{authorcolumns}$ minus the
% remaining number of authors is even, then layout those authors in
% the columns that are in the middle. Otherwise, just lay out the
% authors.
%    \begin{macrocode}
    \begingroup
    \itshape
    \count@\usenix@numauthors
    \divide\count@\authorcolumns
    \multiply\count@\authorcolumns
    \count\z@\authorcolumns
    \advance\count\z@-\usenix@numauthors
    \advance\count\z@\count@
    \ifodd\count\z@
        \count\z@\authorcolumns
    \else
        \divide\count\z@\tw@
    \fi
    \count\tw@\count@
    \count4\z@
    \hb@xt@\textwidth{\hfil
    \loop\ifnum\count4<\authorcolumns
        \ifnum\count4>\z@
            \hskip\authorsep\relax
            \fi
        \ifnum\count\z@>\z@
            \usenix@layoutauthorcolumn{\count4}\usenix@numauthors
            \advance\count\z@\m@ne
        \else
            \usenix@layoutauthorcolumn{\count4}{\count\tw@}%
            \advance\count\tw@\@ne
        \fi
        \advance\count4\@ne
    \repeat
    \global\usenix@currauthor\count\tw@
    \hfil}%
    \ifnum\usenix@currauthor<\usenix@numauthors
        \par\vskip\authorskip\relax
        \hb@xt@\textwidth{\hfil
        \loop\ifnum\usenix@currauthor<\usenix@numauthors
            \begin{tabular}[t]{c}%
            \usenix@authorname\\
            \usenix@institutionname
            \end{tabular}%
            \global\advance\usenix@currauthor\@ne
            \ifnum\usenix@currauthor<\usenix@numauthors
                \hskip\authorsep\relax
            \fi
        \repeat
        \hfil}
    \fi
    \endgroup
}
%    \end{macrocode}
% \end{macro}
%
% \begin{macro}{\usenix@layoutauthorcolumn}
% At last, the code for handling a single column. The first argument
% is the starting point, the second is an additional author to place
% in the column for the final row. The column is layed out by starting
% at the first argument and then striding through the authors by
% |\authorcolumns| until |\count@| is reached. Then the second author
% is layed out, assuming it is in range.
%    \begin{macrocode}
\newcommand\usenix@layoutauthorcolumn[2]{%
    \global\usenix@currauthor#1
    \begin{tabular}[t]{c}%
    \usenix@l@youtauthorcolumn
    \global\usenix@currauthor#2
    \ifnum\usenix@currauthor<\usenix@numauthors
        \\[\authorskip]%
        \usenix@authorname\\
        \usenix@institutionname
    \fi
    \end{tabular}%
}
\newcommand\usenix@l@youtauthorcolumn{%
    \usenix@authorname\\
    \usenix@institutionname
    \global\advance\usenix@currauthor\authorcolumns
    \ifnum\usenix@currauthor<\count@
        \\[\authorskip]%
        \expandafter\usenix@l@youtauthorcolumn
    \fi
}
%    \end{macrocode}
% \end{macro}
%
% Add the newline helpers.
% \changes{v1.2.8}{2011/01/29}{Support multiple line author names}
% \begin{macro}{\authornewline}
% \begin{macro}{\institutionnewline}
%    \begin{macrocode}
\newcommand*\authornewline{%
    \unskip
    \endgroup
    \tabularnewline
    \begingroup
    \upshape
    \ignorespaces
}
\newcommand*\institutionnewline{%
    \unskip
    \tabularnewline
    \ignorespaces
}
%    \end{macrocode}
% \end{macro}
% \end{macro}
%
% \subsection{Setting up the section headers}
% USENIX requires that the section heading be in 12~pt \textbf{bold}.
% There is no requirement on the others, so just make them smaller.
% \changes{v1.2.1}{2009/06/04}{Make subsubsection bold}
% \changes{v1.2.1}{2009/06/04}{Use 11~pt for subsection}
% \changes{v1.2.3}{2009/06/16}{Additional institution rows in italics}
%    \begin{macrocode}
\titleformat\section{\bfseries\large}{\thetitle}{1em}{}{} 
\titleformat\subsection{\bfseries\fontsize\@xipt{13}\selectfont}{\thetitle}{1em}{}{} 
\titleformat\subsubsection{\bfseries}{\thetitle}{1em}{}{}
\endinput
%    \end{macrocode}
% \endinput
% vim: set ts=4 sts=4 expandtab:
