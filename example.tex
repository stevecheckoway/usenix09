% Use titlecompat for \maketitle spacing from USENIX's usenix.sty
% \documentclass[twocolumn,footasend,titlecompat]{usenix09}
\documentclass[twocolumn,footasend]{usenix09}
\usepackage{url}
\usepackage{listings}
\usepackage{lipsum} % Just for example text

\author{Author One}{Institution One}
\author[\texttt{here.is@n.email} Note that email addresses are suppose
to appear on the front page. It's unclear \emph{where} exactly they
are supposed to appear. Footnotes like this?]{Author Two}{Institution Two}
\author{Name in Roman}{Institution in Italics}
\author{Author Four}{Institution Four}
\author{Look At The}{Automatic Author Layout}
\author{It Sure}{Is Fancy}
\title{A 14 pt, bold title}

\begin{document}
\maketitle
\begin{abstract}
An example document showing how easy it is to use this class to
produce USENIX documents.
\end{abstract}

\section{A 12 pt bold section heading}
Standard 10~pt text. Note that instead of footnotes, \verb!\footnote!
produces an endnote when using the \verb!footasend! class
option.\footnote{Fancy, huh?}

\subsection{An 11 pt bold subsection heading}
Does this conform to USENIX's style? Who knows, it sure looks better
than making the sizes the same! The style file for 2005 seems to
indicate that smaller is okay.%
\footnote{http://www.usenix.org/events/samples/usenix.sty}

\subsubsection{A 10 pt bold subsubsection heading}
See Figure~\ref{fig:example}.

\section{Wasting space}
\lipsum

\begin{figure}
\centering
This is where I would normally put a glorious figure, but I have none
to share.
\caption{Note that the caption is typeset in 9~pt as mandated by the
instructions.}
\label{fig:example}
\end{figure}

\section*{An unnumbered section.}
Typically used for acknowledgments and the like.

\appendix
\section{Example}
If you were wondering how this was written, wonder no more. The
complete source is in Listing~\ref{lst:example}.
\lstinputlisting[language={[LaTeX]TeX},float=*,caption=The complete
source for this example file.,label=lst:example,
basicstyle=\footnotesize]{\jobname}
\end{document}
